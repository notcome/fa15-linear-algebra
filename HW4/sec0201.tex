\gotosection{2}{1}
\subsection{The main algorithm: row reduction}

\begin{exercise}{2}
  \begin{enumerate}
    \item $[A|\Vect{b}] = \xmat{0&3&-1&0\\-2&1&2&0\\1&0&-5&0}
      \rightarrow \rops{R_1: R_3\\R_3: R_1}\xmat{1&0&-5&0\\-2&1&2&0\\0&3&-1&0}
      \rightarrow \rops{R_2: 2R_1+R_2}\\ \xmat{1&0&-5&0\\0&1&-8&0\\0&3&-1&0}
      \rightarrow \rops{R_3: -3R_2+R_3}\xmat{1&0&-5&0\\0&1&-8&0\\0&0&23&0}
      \rightarrow \rops{R_3: 1/23R_3}\\\xmat{1&0&-5&0\\0&1&-8&0\\0&0&1&0}
      \rightarrow \rops{R_1: R_1+5R_3\\R_2: R_2+8R_3}\xmat{1&0&0&0\\0&1&0&0\\0&0&1&0}
      = [\widetilde{A}|\widetilde{b}]$
    \item $[A|\Vect{b}] = \xmat{2&3&-1&1\\0&-2&1&2\\1&0&-2&-1}
      \rightarrow \rops{R_1: R_3\\R_3: R_1}\xmat{1&0&-2&-1\\0&-2&1&2\\2&3&-1&1}
      \rightarrow \rops{R_3: -2R_1 + R_3}\\\xmat{1&0&-2&-1\\0&-2&1&2\\0&3&3&3}
      \rightarrow \rops{R_2: -1/2R_2}\xmat{1&0&-2&-1\\0&1&-1/2&-1\\0&3&3&3}
      \rightarrow \rops{R_3: -3R_2+R_3}\\ \xmat{1&0&-2&-1\\0&1&-1/2&-1\\0&0&9/2&6}
      \rightarrow \rops{R_3: 2/9R_3}\xmat{1&0&-2&-1\\0&1&-1/2&-1\\0&0&1&4/3}
      \rightarrow \rops{R_1: R_1 + 2R_3\\R_2: R_2 + 1/2R3} \\
                  \xmat{1&0&0&5/3\\ 0&1&0&-1/3\\ 0&0&1&4/3}
      = [\widetilde{A}|\widetilde{b}]$
  \end{enumerate}
\end{exercise}

\begin{exercise}{3}
  \begin{enumerate}
    \item $A = \xmat{1&2&3 \\ 4&5&6} \rightarrow
      \rops{R_2: -4R_1 + R_2} \xmat{1&2&3 \\ 0&-3&-6} \rightarrow
      \rops{R_2: -1/3R_2} \xmat{1&2&3 \\ 0&1&2} \rightarrow
      \rops{R_1: R_1 - 2R_2} \xmat{1&0&-1 \\ 0&1&2} = \widetilde{A}$
      
    \item $A = \xmat{1&-1&1 \\ -1&0&2 \\ -1&1&1} \rightarrow
      \rops{R_2: R_1 + R_2\\ R_3: R_1 + R_3} \xmat{1&-1&1 \\ 0&-1&3 \\ 0&0&1} \rightarrow
      \rops{R_2: -R_2} \xmat{1&-1&1 \\ 0&1&-3 \\ 0&0&1} \rightarrow
      \rops{R_1: R_1 + R_2} \xmat{1&0&-2 \\ 0&1&-3 \\ 0&0&1} \rightarrow
      \rops{R_1: R_1 + 2R_3\\ R_2: R_2 + 3R_3} \xmat{1&0&0 \\ 0&1&0 \\ 0&0&1}
      = \widetilde{A}$
      
    \item $A = \xmat{1&2&3&5 \\ 2&3&0&-1 \\ 0&1&2&3} \rightarrow
        \rops{R_2: -2R_1 + R_2} \xmat{1&2&3&5 \\ 0&-1&-6&-11 \\ 0&1&2&3} \rightarrow
        \rops{R_2: R_3 \\ R_3: R_2} \\ \xmat{1&2&3&5 \\ 0&1&2&3 \\ 0&-1&-6&-11} \rightarrow
        \rops{R_1: R_1 - R_2 \\ R_3: R_2 + R_3}
          \xmat{1&0&-1&-1 \\ 0&1&2&3 \\ 0&0&-4&-8} \rightarrow
        \rops{R_3: -1/4R_3} \\ \xmat{1&0&-1&-1 \\ 0&1&2&3 \\ 0&0&1&2} \rightarrow
        \rops{R_1: R_1 + R_3 \\ R_2: R_2 - 2R_3}
          \xmat{1&0&0&1 \\ 0&1&0&-1 \\ 0&0&1&2} = \widetilde{A}$
          
    \item $A = \xmat{1&3&-1&4 \\ 1&2&1&2 \\3&7&1&9} \rightarrow
      \rops{R_2: -1R_1 + R_2\\ R_3: -3R_1 + R_3}
        \xmat{1&3&-1&4 \\ 0&-1&2&-2 \\ 0&-2&4&-3} \rightarrow
      \rops{R_2: -R_2} \\ \xmat{1&3&-1&4 \\ 0&1&-2&2 \\ 0&-2&4&-3} \rightarrow
      \rops{R_1: R_1 - 3R_2\\ R_3: 2R_2 + R_3}
        \xmat{1&0&5&-2 \\ 0&1&-2&2 \\ 0&0&0&1} = \widetilde{A}$
        
    \item $A = \xmat{1&1&1&1 \\ 2&-3&3&3 \\ 1&-4&2&2} \rightarrow
      \rops{R_2: -2R_1 + R_2\\ R_3: -R_1 + R_3}
        \xmat{1&1&1&1 \\ 0&-5&1&1 \\ 0&-5&1&1} \rightarrow
      \rops{R_2: -1/5R_2} \\
        \xmat{1&1&1&1 \\ 0&1&-1/5&-1/5 \\ 0&-5&1&1} \rightarrow
      \rops{R_1: R_1 - R_2\\ R_3: 5R_2 + R_3}
        \xmat{1&0&6/5&6/5 \\ 0&1&-1/5&-1/5 \\ 0&0&-1&-1} \rightarrow
      \rops{R_3: -R_3} \\
        \xmat{1&0&6/5&6/5 \\ 0&1&-1/5&-1/5 \\ 0&0&1&1} \rightarrow
      \rops{R_1: R_1 - 6/5R_3\\ R_2: R_2 + 1/5R_3}
        \xmat{1&0&0&0 \\ 0&1&0&0 \\ 0&0&1&1} = \widetilde{A}$
  \end{enumerate}
\end{exercise}

\begin{exercise}{5}
  \begin{enumerate}
    \item Multiplying a row by a nonzero number $c$.
    
        Such operation can be undone by multiplying the result row with $1/c$, since $\forall x \in \mathbb{R}$, $(1/c)cx = ((1/c)c)x = x$, namely $1/c$ is $c$'s inverse with $1$ being the identity providing that $c \neq 1$, and multiplication over real numbers is associative. Since $1/c /neq = 0$, this undoing is also a row operation.
    
    \item Adding a multiple $c$ of a row $r_0$ onto another row $r_x$.
    
        Such operation can be undone by adding the result row with $-cr_0$, since $\forall c \in \mathbb{R}$, $-cr_0 + cr_0 + r_x = (-cr_0 + cr_0) + r_x = (-c+c)r_0 + r_x = r_x$, namely $-c$ is $c$'s inverse with $0$ being the identity, addition over vectors of real numbers is associative, and scaling of vectors of real numbers is distributive. Since $-cr_0$ is a multiple of $r_0$, this undoing is also a row operation.
        
    \item Exchanging two rows.
    
    Exchanging them back. This is also a row reduction.
  \end{enumerate}
  
  Therefore, any row operation can be undone by another row reduction. \rQED
\end{exercise}

\begin{exercise}{8}
  Let the number of rows of $A$ be $n$. Since each row has at most one pivot, $\widetilde{A}$ has at most $n$ pivots.
  
  Supposing that $\widetilde{A}$ has $n$ pivots. Since each column has at most one pivot, the last column of $\widetilde{A}$ has a pivot. Since each row must have exactly one pivot, the last row of $\widetilde{A}$ has a pivot. Since the pivotal $1$ of a lower row is always to the right of the pivotal $1$ of a higher row, and the $n-1$-th rows have $n-1$ pivots, the last column of the last row of $\widetilde{A}$ must be the pivot. Since the $n-1 \times n-1$ matrix of the upper left part of $\widetilde{A}$ has the rest $n-1$ pivots, with all those requirements still applied, the rightmost column's bottommost row is a pivot. Inductively, one can prove that all pivots lie on the orthogonal of $\widetilde{A}$. Since in every column that contains a pivotal 1, all other entries are 0, $\widetilde{A}$'s rest entries must be 0. Therefore, $\widetilde{A}$ is an identity matrix.
  
  Supposing that $\widetilde{A}$ has less than $n$ pivots. Since in every row, the first nonzero entry is $1$, there must be some rows consisting only 0, otherwise there will be $n$ pivots. Then, since any rows consisting entirely of $0$'s are at the bottom, such $\widetilde{A}$'s bottom row must be all $0$. Therefore, $\widetilde{A}$ is either the identity or the last row is a row of zeros. \rQED
\end{exercise}