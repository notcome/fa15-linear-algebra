\gotosection{2}{3}
\subsection{Matrix inverses and elementary matrices}

\begin{exercise}{2}
\begin{enumerate}
\item $\xmat{1/6&5/54\\-1/6&1/54}$

\item Inverse doesn't exist, as $[A|I]$ is reduced to $\xmat{1&3&1&0\\0&0&-3&1}$.

\item $\xmat{2&-1&-3\\-2&2&3\\1&-1&-1}$

\item Inverse doesn't exist, as this is not a square matrix.

\item $\xmat{1/7 & -1/2 & 1/14 \\ 4/21 & 5/6 & -1/14 \\ -4/21 & 1/6 & 1/14}$

\item $\xmat{0&1&-1 \\ 1&-2& 1&-1&1}$

\item $\xmat{4 & -6 & 4 & 1 \\ -6 & 14 & -11 & 3 \\ 4 & -11 & 10 & -3 \\ -1 & 3 & -3 & 1}$
\end{enumerate}
\end{exercise}

\begin{exercise}{3}
\begin{enumerate}
\item Suppose an matrix $A \in M_{m,n}(\mathbb{R})$. If $A$ has an inverse matrix, then $T_A$ must be bijective, or for any $\Vect{b} \in \mathbb{R}^n$, $A\Vect{x}=\Vect{b}$ has a unique solution. If $m < n$, then $\widetilde{A}$ has at least one column without a pivotal 1, so there exist infinitely many solutions, namely $T_A$ is not injective. If $m > n$, there must exist a row-reduced vector $\widetilde{\mathbf{b}}$ that contains a pivotal 1---since row operations are invertible, we can always find a $\Vect{b}$ for any given $\widetilde{\mathbf{b}}$---there exists no solution, namely $T_A$ is not surjective. Only when $m = n$ could $A$ be reduced to an identity matrix and therefore has a unique solution for every possible $\widetilde{\mathbf{b}}$. Hence only squre matrices have inverses. \rQED

\item $A = \xmat{1&0&0\\0&1&0}, B = \xmat{1&0\\0&1\\0&0}$.
\end{enumerate}
\end{exercise}

\begin{exercise}{5}
Label them as $A, B, C$. $A+C$ and $B+2C$ gives $4x+4z=2, 4x+3y=3$, therefore, $x = (1-2z)/2$. Substituting $x$ with $z$ gives $y = (1+4z)/3$. Substituting $x$ and $y$ with $z$ into $A$ gives $z = 1/8$, then $x = 3/8$, $y = 1/2$.

\begin{enumerate}
\item Echelon form of this linear system is
$\xmat{1 & 0 & 0 & 3/8\\
       0 & 1 & 0 & 1/2\\
       0 & 0 & 1 & 1/8}$. Therefore, $x = 3/8$, $y = 1/2$, $z=1/8$.

\item $A^{-1} = \xmat{3/16 & 1/4 & -1/16\\
                      -1/4 & 0   & 3/4  \\
                      1/16 & -1/4 & 5/16}$. $A^{-1}\xmat{1\\1\\1} = \xmat{3/8\\1/2\\1/8}$.
\end{enumerate}
\end{exercise}

\begin{exercise}{8}
\begin{enumerate}
\item \begin{enumerate}
        \item Multiply the second row by 3.
        \item Exchange the second and the third row.
        \item Add a multiple 2 of the first row to the third row.
      \end{enumerate}

\item \begin{enumerate}
        \item $\xmat{1&0&-1\\6&3&3\\0&1&2}$.
        \item $\xmat{1&0&-1\\0&1&2\\2&1&1}$.
        \item $\xmat{1&0&-1\\2&1&1\\2&1&0}$.
      \end{enumerate}

\item \begin{enumerate}
        \item Multiply the second column by 3.
        \item Exchange the second and the third column.
        \item Add a multiple 2 of the first column to the third column.
      \end{enumerate}

\item \begin{enumerate}
        \item $\xmat{1&0&-1\\2&3&1\\0&3&2}$.
        \item $\xmat{1&-1&0\\2&1&1\\0&2&1}$.
        \item $\xmat{1&0&1\\2&1&5\\0&1&2}$.
      \end{enumerate}
\end{enumerate}
\end{exercise}

\begin{exercise}{11}
Column operations applied to a matrix $A$ can be seen as the transpose of the matrix generated by applying the row operations with the same parameters to $A^{\top}$. Here ``the same parameters" means the same index of row/column to multiply, the same multiplication constant, the same indexes of the two row/columns to exchange, or the same multiple of the same index of row/column to be added to another same index of row/column.

Given any elementary matrix $E$, $AE^{\top} = (EA^{\top})^{\top}$. Since $E^{\top}$ is also an elementary matrix, column operations can be achieved by multiplication on the right by elementary matrices. Note that if $E$ is of type 1 or 3, its transpose is the same as itself; but if $E = E_2(i,j,x)$ for some $i, j, x$, its transpose will be $E_2(j,i,x)$, which means that $AE_2(i,j,x)$ will add the multiple $x$ of the $i$th column of $A$ to its $j$th column, rather than adding that multiple of the $j$th column of $A$ to its $i$th column. \rQED
\end{exercise}

\begin{exercise}{12}
\begin{enumerate}
\item Show $E_1(i,x)E_1(i,1/x) = E_1(i,1/x)E_1(i,x) = I$.

\Proof\ For $E_1(i,x)E_1(i,1/x)$, the $i$th-row of $E_1(i,1/x)$ will be multiplied by $x$, then its $i,i$ entry will become $x \cdot (1/x) = 1$, therefore the result is an identity matrix. For $E_1(i,1/x)E_1(i,x)$, the $i$th-row of $E_1(i,x)$ will be multiplied by $1/x$, then its $i,i$ entry will become $(1/x) \cdot x = 1$, therefore the result is an identity matrix. This shows that $E_1(i,1/x)$ is the inverse matrix of $E_1(i,x)$, and that $E_1(i,x)$ is invertible. \rQED

\item Show $E_2(i,j,x)E_2(i,j,-x) = E_2(i,j,-x)E_2(i,j,x) = I$.

\Proof\ Consider $E_2(i,j,-x)E_2(i,j,x)I$, which can be seen as applying two row operations to the identity matrix. Let $r_k$ be the $k$-th row of $I$. After the first operation $E_2(i,j,x)$, the $i$th row becomes $xr_j + r_i$; after the second operation $E_2(i,j,-x)$, the $i$th row of the identity matrix becomes $xr_j + r_i - xr_j = r_i$. Since all other rows of the identity matrix keeps invariant during this two operations, after the two operations the identity itself keeps invariant, namely $E_2(i,j,-x)E_2(i,j,x)I = I$, or $E_2(i,j,-x)E_2(i,j,x)=I$. From what we just proved we can get $E_2(i,j,x)E_2(i,j,-x) = I$ by seeing $-x$ as $y$ and $x$ as $-y$. Then, $E_2(i,j,-x)$ is both the left and right inverse matrix of $E_2(i,j,x)$, namely $E_2(i,j,-x)$ is the inverse of $E_2(i,j,x)$ and the later is invertible. \rQED

\item Show $E3(i,j)E3(i,j) = I$.

\Proof\ Since $E3(i,j)I = E3(i,j)$, $E3(i,j)$ is just the identity matrix with its $i,j$ rows exchanged. $E3(i,j)E3(i,j)$ exchanges these two rows back, so the product matrix is just the identity matrix. This shows that $E3(i,j)E3(i,j) = I$ and that $E3(i,j)$ is the inverse of itself. \rQED
\end{enumerate}
\end{exercise}