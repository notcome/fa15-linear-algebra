\gotosection{0}{4}
\subsection{Functions}

\begin{exercise}{9}
  \begin{enumerate}
    \item $\begin{aligned}[t]
            (f \circ g \circ h)(a) &= f(g(h(a))) = f(g(h(3))) \\
                                   &= f(g(-3+2)) = f(g(-1)) \\
                                   &= f(3 \times (-1)) = f(-3) \\
                                   &= (-3)^2-1 = 8
          \end{aligned}$
    \item $\begin{aligned}
            (f \circ g \circ h)(a) &= f(g(h(a))) = f(g(h(1))) \\
                                   &= f(g(1-3)) = f(g(-2)) \\
                                   &= f(-2-3) = f(-5) \\
                                   &= (-5)^2 = 25
          \end{aligned}$
  \end{enumerate}
\end{exercise}

\begin{exercise}{10}
  \begin{enumerate}
    \item Let the function \FunSS{f}{B}{C} and \FunSS{g}{A}{B} be onto.
          Then the composition $f \circ g$ is onto.

    \Proof{} For every $c \in C$: since \FunSS{f}{B}{C} is surjective, there
    exists $b \in B$ such that $f(b) = c$; since \FunSS{g}{A}{B} is surjective,
    there exists $a \in A$ such that $g(a) = b$. Therefore, for every $c \in C$,
    there exists $a \in A$ such that $(f \circ g)(a) = f(g(a)) = c$.
    Thus, $f \circ g$ is surjective, or onto. \QED

    \item Let the function \FunSS{f}{B}{C} and \FunSS{g}{A}{B} be one to one.
          Then the composition $f \circ g$ is one to one.

    \Proof{} For every $c \in C$ such that there exists $a \in A$ such that
    $f(g(a)) = c$, suppose there do exist $x, y \in A$ such that both $f(g(x)) = c$
    and $f(g(y)) = c$. Since $f$ is injective, $g(x) = g(y)$. Since $g$ is injective
    , $x = y$. Therefore, there is at most one $x$ such that $(f \circ g)(x) =
    f(g(x)) = c$. Thus, $f \circ g$ is injective, or one to one. \QED
  \end{enumerate}
\end{exercise}
