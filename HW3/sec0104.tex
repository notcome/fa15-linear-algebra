\gotosection{1}{4}
\subsection{The geometry of $\mathbb{R}^n$}

\begin{exercise}{3}
  \begin{enumerate}
    \item Let $\Vect{x} = \begin{bmatrix}0\\1\\4\end{bmatrix}$,
          $\frac{\Vect{x}}{|\Vect{x}|}=
           \frac{\Vect{x}}{\sqrt{17}}=
           \begin{bmatrix}0\\\frac{1}{\sqrt{17}}\\\frac{4}{\sqrt{17}}\end{bmatrix}$

    \item Let $\Vect{x} = \begin{bmatrix}-3\\7\end{bmatrix}$,
          $\frac{\Vect{x}}{|\Vect{x}|}=
           \frac{\Vect{x}}{\sqrt{58}}=
           \begin{bmatrix}-\frac{3}{\sqrt{58}}\\\frac{7}{\sqrt{58}}\end{bmatrix}$

    \item Let $\Vect{x} = \begin{bmatrix}\sqrt{2}\\-2\\-5\end{bmatrix}$,
          $\frac{\Vect{x}}{|\Vect{x}|}=
           \frac{\Vect{x}}{\sqrt{31}}=
           \begin{bmatrix}
             \sqrt{\frac{2}{31}} \\
             -\frac{2}{\sqrt{31}} \\
             -\frac{5}{\sqrt{31}}
           \end{bmatrix}$
  \end{enumerate}
\end{exercise}

\begin{exercise}{5}
  \begin{enumerate}
    \item $cos(\theta) = \frac{1}{\sqrt{3}}$, $\theta = \arccos{1}{\sqrt{3}}$.
    \item $cos(\theta) = 0$, $\theta = \frac{\pi}{2}$.
  \end{enumerate}
\end{exercise}

\begin{exercise}{7}
  \begin{enumerate}
    \item $\det(M) = 1$, $M^{-1} = \begin{xmatrix}0&1\\-1&2\end{xmatrix}$.
    \item $\det(M) = 0$, $M^{-1}$ doesn't exist.
    \item $\det(M) = ad$, if $a, d \neq 0$, $M^{-1} = \frac{1}{ad}\begin{xmatrix}d&-b\\0&a\end{xmatrix}$.
    \item $\det(M) = 0$, $M^{-1}$ doesn't exist.
  \end{enumerate}
\end{exercise}

\begin{exercise}{8}
  \begin{enumerate}
    \item -4
    \item $adf$
    \item $adg - cbg$
  \end{enumerate}
\end{exercise}

\begin{exercise}{10}
  \begin{enumerate}
    \item This can be proven inductively: $k = 1$, $|A^1| = |A| = |A|^1$. $k = 2$, $|A^2| = |A \cdot A| \leq |A||A| = |A|^2$, from Cauchy-Schwarz's inequality. $k > 2$, $|A^k| = |A^{k-1} \cdot A| \leq |A^{k-1}||A| \leq |A|^k$. \rQED
          $$|A^3| = |\begin{xmatrix}7&10\\5&7\end{xmatrix}| = \sqrt{223}$$
          $$|A|^3 = (\sqrt{7})^3 = 7\sqrt{7}$$
          
    \item $|\Vect{u} \cdot \Vect{v}| = |\Vect{u}||\Vect{v}|$, because they are in opposite direction, namely $\Vect{u}$ is on the line spanned from $\Vect{v}$.
          
          $|\Vect{u} \cdot \Vect{w}| < |\Vect{u}||\Vect{w}|$, because the two vectors are not in the same or opposite directions, namely one is not on the line spanned from the other.
          
          $|\Vect{u} \cdot \Vect{v}| = 2 + 8 + 18 = 28 = \sqrt{14} \cdot \sqrt{56} = |\Vect{u}||\Vect{v}|$.
          
          $|\Vect{u} \cdot \Vect{w}| = 2 + 18 = 20$, $|\Vect{u}||\Vect{w}| = \sqrt{14} \cdot \sqrt{40} = 4\sqrt{35} > 20 = |\Vect{u} \cdot \Vect{w}|$.
    
    \item Let $\Vect{u} = \begin{bmatrix}-v_2\\v_1\end{bmatrix}$, namely rotating $\Vect{v}$ by $\frac{\pi}{2}$ counterclockwise. $\Vect{u} \cdot \Vect{w} = -v_2w_1 + v_1w_2 < 0$. Therefore, the angle between $\Vect{u}$ and $\Vect{w}$ is larger than $\frac{\pi}{2}$, which means $\Vect{w}$ lies clockwise from $\Vect{v}$.
    
    \item Let $\theta$ be the angle between $\Vect{v}$ and $\Vect{w}$. $\cos(\theta) = \frac{\Vect{v} \cdot \Vect{w}}{|\Vect{v}||\Vect{w}|}$, $\cos(\theta)|\Vect{v}||\Vect{w}| = 42$.
    
          If $\cos(\theta) = 1$, $|\Vect{v}||\Vect{w}| = 42$, or $|\Vect{w}| = 3\sqrt{14}$. This is the shortest $\Vect{w}$. As $\cos(\theta)$ approaches $0$, $|\Vect{w}|$ will approach $\infty$, so there is no longest $\Vect{w}$.
  \end{enumerate}
\end{exercise}

\begin{exercise}{13}
  First, we can prove that for any
    $\Vect{a} = \begin{xmatrix}a_1\\a_2\\a_3\end{xmatrix},
     \Vect{b} = \begin{xmatrix}b_1\\b_2\\b_3\end{xmatrix}$ and $x, y \in \mathbb{R}$, $x\Vect{a} \times y\Vect{b} = xy(\Vect{a} \times \Vect{b})$:
  
  $$x\Vect{a} \times y\Vect{b} =
    \begin{xmatrix}
       xa_2yb_3 - xa_3yb_2 \\
      -xa_1yb_3 + xa_3yb_1 \\
       xa_1yb_2 - xa_2yb_1
    \end{xmatrix} = xy
    \begin{xmatrix}
       a_2b_3 - a_3b_2 \\
      -a_1b_3 + a_3b_1 \\
       a_1b_2 - a_2b_1
    \end{xmatrix} = xy(\Vect{a} \times \Vect{b})$$
    
  For two vectors $\Vect{v}_1$, $\Vect{v}_2$ pointing in the same direction, they can be written as $n_1\Vect{u}$ and $n_2\Vect{u}$, respectively. Let $u_1, u_2, u_3$ be the three components of $\Vect{u}$, we have:

  $$\Vect{v}_1 \times \Vect{v}_2 = (n_1n_2) \Vect{u} \times \Vect{u} = (n_1n_2)
    \begin{xmatrix}
       u_2u_3 - u_3u_2 \\
      -u_1u_3 + u_3u_1 \\
       u_1u_2 - u_2u_1
    \end{xmatrix} = (n_1n_2)\Vect{0} = \Vect{0}$$
    
  \rQED
\end{exercise}

\begin{exercise}{16}
  \begin{enumerate}
    \item $|Area| = |\det(\begin{xmatrix}1&5\\2&1\end{xmatrix})| = |1 - 10| = 9$
    \item $|Area| = |\det(\begin{xmatrix}1&5\\2&-1\end{xmatrix})| = |-1 - 10| = 11$
  \end{enumerate}
\end{exercise}

\begin{exercise}{19}
  \begin{enumerate}
    \item $|\Vect{v}_n| = \sqrt{1 + \hdots + 1} = \sqrt{n}$
    \item $\cos{a_n} = \frac{\Vect{v}_n \cdot \Vect{e}_1}{|\Vect{v}_n||\Vect{e}_1|} = \frac{1}{\sqrt{n}}$, so $a_n = \arccos{\frac{1}{\sqrt{n}}}$. $\lim_{n \to \infty} a_n = \arccos{\frac{1}{\infty}} = \frac{\pi}{2}$.
  \end{enumerate}
\end{exercise}

\begin{exercise}{24}
  \def \Set {\Vect{v}^{\bot}}
  \begin{enumerate}
    \item Since $\Vect{0} \cdot \Vect{v} = 0$, $\Vect{0} \in \Set$.
    
          For any $\Vect{x}, \Vect{y} \in \Set$, $(\Vect{x} + \Vect{y}) \cdot \Vect{v} = \Vect{x} \cdot \Vect{v} + \Vect{y} \cdot \Vect{v} = 0 + 0 = 0$. Therefore, $(\Vect{x} + \Vect{y}) \in \Set$.
          
          For $\Vect{x} \in \Set$ and $c \in \mathbb{R}$, $(c\Vect{x}) \cdot \Vect{v} = c(\Vect{x} \cdot \Vect{v}) = c0 = 0$. \rQED
    
    \item $(\Vect{a} - \frac{\Vect{a} \cdot \Vect{v}}{|\Vect{v}|^2}\Vect{v}) \cdot \Vect{v} =
            \Vect{a} \cdot \Vect{v} - \frac{\Vect{a}\cdot\Vect{v}}{|\Vect{v}|^2}\Vect{v} \cdot \Vect{v} =
            \Vect{a} \cdot \Vect{v}(1 - \frac{\Vect{v}^2}{|\Vect{v}|^2}) = 0$. \rQED
            
    \item (b) suggests the existence of $t(\Vect{a})$. Suppose that there exists a distinct $t'(\Vect{a})$ such that $(\Vect{a} + t'(\Vect{a})\Vect{v}) \in \Set$. Then we have:
    
         $$\left\{
         \begin{aligned}
            (\Vect{a} + t(\Vect{a})\Vect{v}) \cdot \Vect{v} = 0 \\
            (\Vect{a} + t'(\Vect{a})\Vect{v}) \cdot \Vect{v} = 0
         \end{aligned} \quad \Rightarrow \quad
         \begin{aligned}
            \Vect{a} \cdot \Vect{v} + t(\Vect{a})\Vect{v}^2 = 0 \\
            \Vect{a} \cdot \Vect{v} + t'(\Vect{a})\Vect{v}^2 = 0
         \end{aligned} \quad \Rightarrow \quad 
         \begin{aligned}
            t(\Vect{a}) = -\frac{\Vect{a} \cdot \Vect{v}}{\Vect{v}^2} \\
            t'(\Vect{a}) = -\frac{\Vect{a} \cdot \Vect{v}}{\Vect{v}^2}
         \end{aligned}
         \right.$$
         
         Since there could be only one possible value for $-\frac{\Vect{a} \cdot \Vect{v}}{\Vect{v}^2}$, $t(\Vect{a}) = t'(\Vect{a})$, namely $t(\Vect{a})$ is unique.
         
        $$\Vect{a} + t(\Vect{a})\Vect{v} = \Vect{a} -\frac{\Vect{a} \cdot \Vect{v}}{\Vect{v}^2}\Vect{v} = \Vect{a} - \frac{\Vect{a} \cdot \Vect{v}}{|\Vect{v}|^2}\Vect{v} = P_{\Set}(\Vect{a})$$
        
        \rQED
  \end{enumerate}
\end{exercise}

\begin{exercise}{26}
  \begin{enumerate}
    \item $$A\begin{xmatrix}x\\y\end{xmatrix} = \begin{xmatrix}x-2y\\3x+4y\end{xmatrix}$$
    
          $$a\begin{pmatrix}x\\y\end{pmatrix} = \arccos{\frac{x^2+xy+4y^2} {\sqrt{10}\sqrt{(x^2+y^2) (x^2+2xy+2y^2)}}}$$
          
    \item Suppose there existing a nonzero vector such that it is rotated by $\frac{\pi}{2}$:
    
          $$\frac{x^2+xy+4y^2} {\sqrt{10}\sqrt{(x^2+y^2) (x^2+2xy+2y^2)}} = \cos(\frac{\pi}{2})$$
          $$x^2+xy+4y^2=0$$
          $$(\frac{1}{2}x+2y)^2=-\frac{3}{4}x^2$$
          
          Because $(\frac{1}{2}x+2y)^2 \geq 0$ and $-\frac{3}{4}x^2 \leq 0$, both sides must equal $0$. Therefore, $x = 0, y = 0$, which contradicts the premise of nonzero vector. As a result, such vector doesn't exist.
  \end{enumerate}
\end{exercise}
