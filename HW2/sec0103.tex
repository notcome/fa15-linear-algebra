\gotosection{1}{3}
\subsection{Linear transformations}

\begin{exercise}{4}
  \begin{enumerate}
    \item $T = \begin{xmatrix}
                 2 & 0 & 0 \\
                 0 & 1 & 0 \\
                 0 & 0 & 1
               \end{xmatrix}$

    \item $T = \begin{xmatrix}
                 0 & 1 & 0 \\
                 1 & 2 & 0 \\
                 1 & 0 & 1
               \end{xmatrix}$
  \end{enumerate}
\end{exercise}

\begin{exercise}{8}
  The $i$th column of $[T]$ is $T(\Vect{e}_i)$.

  \begin{enumerate}
    \item For \FunSS{T}{\mathbb{R}^5}{\mathbb{R}^6}, $[T]$ has 5 columns. Therefore,
          5 questions need answered, and they are $T(\Vect{e}_1), \hdots, T(\Vect{e}_5)$.

    \item For \FunSS{T}{\mathbb{R}^6}{\mathbb{R}^5}, $[T]$ has 6 columns. Therefore,
          6 questions need answered, and they are $T(\Vect{e}_1), \hdots, T(\Vect{e}_6)$.

    \item There are infinite answers. For instance, one can ask $T(2\Vect{e}_1), \hdots, T(2\Vect{e}_6)$.
          Then instead of $T$, one will get the columns of $2T$. However, if one
          multiplies the results with $\frac{1}{2}$, he or she can still get the
          original matrix $T$.
  \end{enumerate}
\end{exercise}

\begin{exercise}{9}
  Suppose $T$ is linear. Due to the linearity:

  \begin{align*}
    T(\begin{xmatrix}2 \\ -1 \\ 4\end{xmatrix}) &= T(2\Vect{e}_1 - \Vect{e}_2 + 4\Vect{e}_3) \\
                                                &= 2T(\Vect{e}_1) - T(\Vect{e}_2) + 4T(\Vect{e}_3) \\
                                                &= 2\begin{xmatrix}2 \\ 1 \\ 1\end{xmatrix} -
                                                    \begin{xmatrix}1 \\ 2 \\ 1\end{xmatrix} +
                                                   4\begin{xmatrix}1 \\ 0 \\ 1\end{xmatrix}
                                                &=  \begin{xmatrix}7 \\ 0 \\ 5\end{xmatrix}
                                                \neq\begin{xmatrix}7 \\ 0 \\ 4\end{xmatrix}
  \end{align*}

  Therefore $T$ is not linear.
\end{exercise}

\begin{exercise}{13}
  \begin{enumerate}
    \item Nonsense.
    \item \FunSS{C \circ B}{\mathbb{R}^m}{\mathbb{R}^n}
    \item \FunSS{A \circ C}{\mathbb{R}^k}{\mathbb{R}^m}
    \item \FunSS{B \circ A \circ C}{\mathbb{R}^k}{\mathbb{R}^k}
    \item Nonsense.
    \item Nonsense.
    \item \FunSS{B \circ A}{\mathbb{R}^n}{\mathbb{R}^k}
    \item \FunSS{A \circ C \circ B}{\mathbb{R}^m}{\mathbb{R}^k}
    \item \FunSS{C \circ B \circ A}{\mathbb{R}^n}{\mathbb{R}^n}
    \item Nonsense.
  \end{enumerate}
\end{exercise}

\begin{exercise}{18}
\end{exercise}

\begin{exercise}{19}
\end{exercise}

\begin{exercise}{20}
\end{exercise}
