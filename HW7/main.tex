\documentclass{article}
\usepackage{amsmath}
\usepackage{amssymb}
\usepackage{calc}
\usepackage{mathtools}
\usepackage{tkz-euclide}

% To reduce the possibility of being reached through Google.
\def \TitleSubject{MATH}
\def \TitleSeqNoA{3}
\def \TitleSeqNoB{1}
\def \TitleSeqId{A}
\def \TitleSeq{\TitleSeqNoA\TitleSeqNoB\TitleSeqId}
\newcommand \MakeTitle[1]{
  \title{\TitleSubject\ \TitleSeq\ Homework\ #1}
  \author{Minsheng Liu}
}

\newcommand \gotosection[2]{
  \setcounter{section}{#1}
  \setcounter{subsection}{{{#2}-1}}
}
\newenvironment{exercise}[1]{
  \setcounter{subsubsection}{{{#1}-1}}
  \subsubsection{}
}{}

% Sub-questions in this book using a, b, c, etc. as counters.
\renewcommand \theenumi{\alph{enumi}}

% Convenient Math constructs.
\newcommand \FunSS[3]{$#1 : #2 \to #3$}
\newcommand \QED{$\square$}
\newcommand \rQED{\hfill\QED}
\newcommand \Proof{\textbf{Proof.}}
\newcommand \Vect[1]{\vec{\mathbf{#1}}}
\newenvironment{xmatrix}{\begin{bmatrix*}[r]}{\end{bmatrix*}}

\usepackage{enumerate}

\MakeTitle{7}

\begin{document}
\maketitle

\begin{enumerate}[1.]
\item
First, we prove that $\vec{w}_i$ is orthogonal to $\vec{w}_1, \hdots, 
\vec{w}_{i-1}$ for $i \leq p$. Rewrite the definition of $w_i$ in terms of
the projection function:
\newcommand \ProjW[1]{\mathrm{proj}_{\vec{w}_{#1}}(\vec{v}_i)}
$$\vec{w}_i = \vec{v}_i - \sum_j\ProjW{j}$$
For some $j < i$:
$$
\begin{aligned}
(\vec{v}_i - \ProjW{j}) \cdot \vec{w}_j &= 0 \\
(\vec{w}_i + \sum_{k \neq j}\ProjW{k}) \cdot \vec{w}_j &= 0 \\
\vec{w}_i \cdot \vec{w}_j + \sum_{k \neq j} \ProjW{k} \cdot \vec{w}_j &= 0
\end{aligned}
$$
From the induction hypothesis we know that $\vec{w}_k \bot \vec{w}_j$ for
$k \neq j, k < i$, so $\ProjW{k} \cdot \vec{w}_j = 0$ for each such $k$.
Therefore, $\vec{w}_i \bot \vec{w}_j$.

Hence we have proven that $\vec{w}_i$ is orthogonal to $\vec{w}_1, \hdots,
\vec{w}_{i-1}$ for $i \leq p$. From this is trivial to see that $\vec{w}_1,
\hdots, \vec{w}_p$ is orthogonal to each other. In other words, they are
linearly independent\footnote{Textbook, \textbf{Proposition 2.4.17}}.

Since $\{\vec{v}_1, \hdots, \vec{v}_p\}$ is a basis of $E$, $E$'s maximal
linearly independent sets should have exactly $p$ elements. Therefore,
$\{\vec{w}_1, \hdots, \vec{w}_p\}$ forms a maximal linearly independent set of
$E$, and it is indeed an orthogonal basis of $E$. \rQED

\item
First, we prove the existence of such decomposition, which is equivalent to the statement that the union of one of  $E$'s basis and one of $E^{\bot}$'s basis is a basis for $R^n$. We only need to prove this is true for some two bases, as the number of vectors in a basis for the same subspace is invariant. We pick an orthogonal basis for $E$ and another one for $E^{\bot}$. Assume that we still need some more vectors to form a basis for $R^n$. Apply the Gram--Schmidt process to these vectors, but ensure we first process vectors of the bases of $E$ and $E^{\top}$. In this way, since vectors of the two bases are orthogonal to each other, the process takes no effect on these vectors---the subtracted vectors, or the projections, are always zero. After the process we get an orthogonal basis for $R^n$, with some of vectors forming a basis for $E$, some for $E^{\bot}$, and some other vectors. But since those other vectors are orthogonal to vectors of a basis for $E$, so they should be elements of $E^{\bot}$. Thus we get a contradiction, and we can safely say that an orthogonal decomposition always exist for vectors in $R^n$.

Next, we prove the uniqueness of such decomposition. Assume that we have $\vec{v} = \vec{x}_1 + \vec{y}_1 = \vec{x}_2 + \vec{y}_2$, where $\vec{x}_1, \vec{x}_2 \in E$, $\vec{y}_1, \vec{y}_2 \in E^{\bot}$, $\vec{x}_1 \neq \vec{x}_2$, and $\vec{y}_1 \neq \vec{y}_2$.
$$
\begin{aligned}
\vec{x}_1 + \vec{y}_1 &= \vec{x}_2 + \vec{y}_2 \\
\vec{x}_1 - \vec{x}_2 &= \vec{y}_2 - \vec{y}_1 \\
\end{aligned}
$$
The intersection of $E$ and $E^{\top}$ is $\{\vec{0}\}$, therefore both sides equal to the zero vector, or $\vec{x}_1 = \vec{x}_2$ and $\vec{y}_1 = \vec{y}_2$, hence another contradiction. As a result, orthogonal decomposition is unique. \rQED
\end{enumerate}
\end{document}
