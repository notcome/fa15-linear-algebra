\documentclass{article}
\usepackage{amsmath}
\usepackage{amssymb}
\usepackage{calc}
\usepackage{mathtools}

% To reduce the possibility of being reached through Google.
\def \TitleSubject{MATH}
\def \TitleSeqNoA{3}
\def \TitleSeqNoB{1}
\def \TitleSeqId{A}
\def \TitleSeq{\TitleSeqNoA\TitleSeqNoB\TitleSeqId}
\newcommand \MakeTitle[1]{
  \title{\TitleSubject\ \TitleSeq\ Homework\ #1}
  \author{Minsheng Liu}
}

% Generate exercise number for quick reference.
\newcommand \exerciseOld[3]{
  \setcounter{section}{#1}
  \setcounter{subsection}{#2}
  % It seems that in LaTeX \subsection starts from 0.
  \setcounter{subsubsection}{{{#3}-1}}
  \subsubsection{}
}

\newcommand \gotosection[2]{
  \setcounter{section}{#1}
  \setcounter{subsection}{#2}
}
\newenvironment{exercise}[1]{
  \setcounter{subsubsection}{{{#1}-1}}
  \subsubsection{}
}{}

% Sub-questions in this book using a, b, c, etc. as counters.
\renewcommand \theenumi{\alph{enumi}}

% Convenient Math constructs.
\newcommand \FunSS[3]{$#1 : #2 \to #3$}
\newcommand \QED{$\square$}
\newcommand \Proof{\textbf{Proof.}}


\MakeTitle{5}

\begin{document}
\maketitle

\gotosection{2}{4}
\subsection{Linear combinations, span, and linear independence}

\def \RRT{\quad\widetilde{}\quad}
\newcommand \spanft[2]{\mathrm{span}(#1,\hdots,#2)}

\begin{exercise}{2}
\begin{enumerate}
\item Since the matrix formed by these three vectors,  $\xmat{1&-2&-1\\2&1&1\\3&2&-1}$, can be row reduced to an identity matrix, they are linearly independent. As a result, they form a basis of $\mathbb{R}^3$. Because $\xmat{1\\2\\3\\} \cdot \xmat{-2\\1\\2} = 6 \neq 0$, these vector do not form an orthogonal set of vectors and therefore not a orthogonal basis. \rQED

\item Since
$$\xmat{4&3&2&4\\2&0&1&1\\1&4&4&2} \RRT \xmat{1&0&0&2/3\\0&1&0&2/3\\0&0&1&-1/3}$$
, the vector \emph{is} in the first span:
$$\xmat{4\\1\\2} = \frac{2}{3}\xmat{4\\2\\1} + \frac{2}{3}\xmat{3\\0\\4} - \frac{1}{3}\xmat{2\\1\\4}$$
And since
$$\xmat{4&3&5&4\\2&0&1&1\\1&4&4.5&2} \RRT \xmat{1&0&0&23/81\\0&1&0&19/81\\0&0&1&35/81}$$
, the vector \emph{is} also in the second one:
$$\xmat{4\\1\\2} = \frac{23}{81}\xmat{4\\2\\1} + \frac{19}{81}\xmat{3\\0\\4} - \frac{35}{81}\xmat{5\\1\\4.5}$$
\end{enumerate}
\end{exercise}

\begin{exercise}{3}
Let $\Vect{v}_1 = \xmat{1\\1}, \Vect{v}_2 = \xmat{1,-1}$, then their corresponding normalized vectors are $\Vect{v}'_1 = \xmat{1/\sqrt{2}\\1/\sqrt{2}}, \Vect{v}'_2 = \xmat{1/\sqrt{2}\\-1/\sqrt{2}}$. $\Vect{v}'_1$ and $\Vect{v}'_2$ are linearly independent to each other: for any $\Vect{w} = a\Vect{v}_1 + b\Vect{v}_2 \in \mathbb{R}^2$, with $\Vect{v}'_1$ and $\Vect{v}'_2$ $\Vect{w}$ can and only can be written as $\sqrt{2}a\Vect{v}'_1 + \sqrt{2}b\Vect{v}'_2$. $\Vect{v}'_1 \cdot \Vect{v}'_2 = 0$, so they are orthogonal to each other. Hence $\Vect{v}'_1$ and $\Vect{v}'_2$ creates a orthonormal basis for $\mathbb{R}^2$.
\end{exercise}

\begin{exercise}{5}
\def \TheSpan{\spanft{\Vect{v}_1}{\Vect{v}_k}}

$\Vect{0} = 0\Vect{v}_1 + \hdots + 0\Vect{v}_k$, therefore, $\Vect{0} \in \TheSpan$. For any $\Vect{u}, \Vect{w} \in \TheSpan$, $\Vect{u} = \sum_{i=1}^ka_i\Vect{v}_i, \Vect{w} = \sum_{i=1}^kb_i\Vect{v}_i$, $\Vect{u} + \Vect{w} = \sum_{i=1}^k(a_i+b_i)\Vect{v}_i \in \TheSpan$. For any $\Vect{z} \in \TheSpan$, $\Vect{z} = \sum_{i=1}^kn_i\Vect{v}_i$, $c\Vect{z} = c\sum_{i=1}^kn_i\Vect{v}_i = \sum_{i=1}^kcn_i\Vect{v}_i \in \TheSpan$. Therefore, $\TheSpan$ is a subspace of $\mathbb{R}^n$. \rQED

Assuming there exists a subspace $V$ of $\mathbb{R}^n$ such that $V \subset \TheSpan$ and $V \neq \TheSpan$. For any $\Vect{u} \in \TheSpan$, since $\Vect{v}_1, \hdots, \Vect{v}_k \in V$ and $\Vect{u} = \sum_{i=1}^ka_i\Vect{v}_i$ for some $a_1, \hdots, a_k$, $\Vect{u} \in V$. Therefore, $V \subset \TheSpan$, $V = \TheSpan$, which contradicts to the assumption. \rQED
\end{exercise}

\end{document}
